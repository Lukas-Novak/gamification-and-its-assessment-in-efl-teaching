% Options for packages loaded elsewhere
\PassOptionsToPackage{unicode}{hyperref}
\PassOptionsToPackage{hyphens}{url}
%
\documentclass[
  man]{apa6}
\usepackage{amsmath,amssymb}
\usepackage{lmodern}
\usepackage{iftex}
\ifPDFTeX
  \usepackage[T1]{fontenc}
  \usepackage[utf8]{inputenc}
  \usepackage{textcomp} % provide euro and other symbols
\else % if luatex or xetex
  \usepackage{unicode-math}
  \defaultfontfeatures{Scale=MatchLowercase}
  \defaultfontfeatures[\rmfamily]{Ligatures=TeX,Scale=1}
\fi
% Use upquote if available, for straight quotes in verbatim environments
\IfFileExists{upquote.sty}{\usepackage{upquote}}{}
\IfFileExists{microtype.sty}{% use microtype if available
  \usepackage[]{microtype}
  \UseMicrotypeSet[protrusion]{basicmath} % disable protrusion for tt fonts
}{}
\makeatletter
\@ifundefined{KOMAClassName}{% if non-KOMA class
  \IfFileExists{parskip.sty}{%
    \usepackage{parskip}
  }{% else
    \setlength{\parindent}{0pt}
    \setlength{\parskip}{6pt plus 2pt minus 1pt}}
}{% if KOMA class
  \KOMAoptions{parskip=half}}
\makeatother
\usepackage{xcolor}
\usepackage{color}
\usepackage{fancyvrb}
\newcommand{\VerbBar}{|}
\newcommand{\VERB}{\Verb[commandchars=\\\{\}]}
\DefineVerbatimEnvironment{Highlighting}{Verbatim}{commandchars=\\\{\}}
% Add ',fontsize=\small' for more characters per line
\usepackage{framed}
\definecolor{shadecolor}{RGB}{248,248,248}
\newenvironment{Shaded}{\begin{snugshade}}{\end{snugshade}}
\newcommand{\AlertTok}[1]{\textcolor[rgb]{0.94,0.16,0.16}{#1}}
\newcommand{\AnnotationTok}[1]{\textcolor[rgb]{0.56,0.35,0.01}{\textbf{\textit{#1}}}}
\newcommand{\AttributeTok}[1]{\textcolor[rgb]{0.77,0.63,0.00}{#1}}
\newcommand{\BaseNTok}[1]{\textcolor[rgb]{0.00,0.00,0.81}{#1}}
\newcommand{\BuiltInTok}[1]{#1}
\newcommand{\CharTok}[1]{\textcolor[rgb]{0.31,0.60,0.02}{#1}}
\newcommand{\CommentTok}[1]{\textcolor[rgb]{0.56,0.35,0.01}{\textit{#1}}}
\newcommand{\CommentVarTok}[1]{\textcolor[rgb]{0.56,0.35,0.01}{\textbf{\textit{#1}}}}
\newcommand{\ConstantTok}[1]{\textcolor[rgb]{0.00,0.00,0.00}{#1}}
\newcommand{\ControlFlowTok}[1]{\textcolor[rgb]{0.13,0.29,0.53}{\textbf{#1}}}
\newcommand{\DataTypeTok}[1]{\textcolor[rgb]{0.13,0.29,0.53}{#1}}
\newcommand{\DecValTok}[1]{\textcolor[rgb]{0.00,0.00,0.81}{#1}}
\newcommand{\DocumentationTok}[1]{\textcolor[rgb]{0.56,0.35,0.01}{\textbf{\textit{#1}}}}
\newcommand{\ErrorTok}[1]{\textcolor[rgb]{0.64,0.00,0.00}{\textbf{#1}}}
\newcommand{\ExtensionTok}[1]{#1}
\newcommand{\FloatTok}[1]{\textcolor[rgb]{0.00,0.00,0.81}{#1}}
\newcommand{\FunctionTok}[1]{\textcolor[rgb]{0.00,0.00,0.00}{#1}}
\newcommand{\ImportTok}[1]{#1}
\newcommand{\InformationTok}[1]{\textcolor[rgb]{0.56,0.35,0.01}{\textbf{\textit{#1}}}}
\newcommand{\KeywordTok}[1]{\textcolor[rgb]{0.13,0.29,0.53}{\textbf{#1}}}
\newcommand{\NormalTok}[1]{#1}
\newcommand{\OperatorTok}[1]{\textcolor[rgb]{0.81,0.36,0.00}{\textbf{#1}}}
\newcommand{\OtherTok}[1]{\textcolor[rgb]{0.56,0.35,0.01}{#1}}
\newcommand{\PreprocessorTok}[1]{\textcolor[rgb]{0.56,0.35,0.01}{\textit{#1}}}
\newcommand{\RegionMarkerTok}[1]{#1}
\newcommand{\SpecialCharTok}[1]{\textcolor[rgb]{0.00,0.00,0.00}{#1}}
\newcommand{\SpecialStringTok}[1]{\textcolor[rgb]{0.31,0.60,0.02}{#1}}
\newcommand{\StringTok}[1]{\textcolor[rgb]{0.31,0.60,0.02}{#1}}
\newcommand{\VariableTok}[1]{\textcolor[rgb]{0.00,0.00,0.00}{#1}}
\newcommand{\VerbatimStringTok}[1]{\textcolor[rgb]{0.31,0.60,0.02}{#1}}
\newcommand{\WarningTok}[1]{\textcolor[rgb]{0.56,0.35,0.01}{\textbf{\textit{#1}}}}
\usepackage{graphicx}
\makeatletter
\def\maxwidth{\ifdim\Gin@nat@width>\linewidth\linewidth\else\Gin@nat@width\fi}
\def\maxheight{\ifdim\Gin@nat@height>\textheight\textheight\else\Gin@nat@height\fi}
\makeatother
% Scale images if necessary, so that they will not overflow the page
% margins by default, and it is still possible to overwrite the defaults
% using explicit options in \includegraphics[width, height, ...]{}
\setkeys{Gin}{width=\maxwidth,height=\maxheight,keepaspectratio}
% Set default figure placement to htbp
\makeatletter
\def\fps@figure{htbp}
\makeatother
\setlength{\emergencystretch}{3em} % prevent overfull lines
\providecommand{\tightlist}{%
  \setlength{\itemsep}{0pt}\setlength{\parskip}{0pt}}
\setcounter{secnumdepth}{-\maxdimen} % remove section numbering
% Make \paragraph and \subparagraph free-standing
\ifx\paragraph\undefined\else
  \let\oldparagraph\paragraph
  \renewcommand{\paragraph}[1]{\oldparagraph{#1}\mbox{}}
\fi
\ifx\subparagraph\undefined\else
  \let\oldsubparagraph\subparagraph
  \renewcommand{\subparagraph}[1]{\oldsubparagraph{#1}\mbox{}}
\fi
\newlength{\cslhangindent}
\setlength{\cslhangindent}{1.5em}
\newlength{\csllabelwidth}
\setlength{\csllabelwidth}{3em}
\newlength{\cslentryspacingunit} % times entry-spacing
\setlength{\cslentryspacingunit}{\parskip}
\newenvironment{CSLReferences}[2] % #1 hanging-ident, #2 entry spacing
 {% don't indent paragraphs
  \setlength{\parindent}{0pt}
  % turn on hanging indent if param 1 is 1
  \ifodd #1
  \let\oldpar\par
  \def\par{\hangindent=\cslhangindent\oldpar}
  \fi
  % set entry spacing
  \setlength{\parskip}{#2\cslentryspacingunit}
 }%
 {}
\usepackage{calc}
\newcommand{\CSLBlock}[1]{#1\hfill\break}
\newcommand{\CSLLeftMargin}[1]{\parbox[t]{\csllabelwidth}{#1}}
\newcommand{\CSLRightInline}[1]{\parbox[t]{\linewidth - \csllabelwidth}{#1}\break}
\newcommand{\CSLIndent}[1]{\hspace{\cslhangindent}#1}
\ifLuaTeX
\usepackage[bidi=basic]{babel}
\else
\usepackage[bidi=default]{babel}
\fi
\babelprovide[main,import]{english}
% get rid of language-specific shorthands (see #6817):
\let\LanguageShortHands\languageshorthands
\def\languageshorthands#1{}
% Manuscript styling
\usepackage{upgreek}
\captionsetup{font=singlespacing,justification=justified}

% Table formatting
\usepackage{longtable}
\usepackage{lscape}
% \usepackage[counterclockwise]{rotating}   % Landscape page setup for large tables
\usepackage{multirow}		% Table styling
\usepackage{tabularx}		% Control Column width
\usepackage[flushleft]{threeparttable}	% Allows for three part tables with a specified notes section
\usepackage{threeparttablex}            % Lets threeparttable work with longtable

% Create new environments so endfloat can handle them
% \newenvironment{ltable}
%   {\begin{landscape}\begin{center}\begin{threeparttable}}
%   {\end{threeparttable}\end{center}\end{landscape}}
\newenvironment{lltable}{\begin{landscape}\begin{center}\begin{ThreePartTable}}{\end{ThreePartTable}\end{center}\end{landscape}}

% Enables adjusting longtable caption width to table width
% Solution found at http://golatex.de/longtable-mit-caption-so-breit-wie-die-tabelle-t15767.html
\makeatletter
\newcommand\LastLTentrywidth{1em}
\newlength\longtablewidth
\setlength{\longtablewidth}{1in}
\newcommand{\getlongtablewidth}{\begingroup \ifcsname LT@\roman{LT@tables}\endcsname \global\longtablewidth=0pt \renewcommand{\LT@entry}[2]{\global\advance\longtablewidth by ##2\relax\gdef\LastLTentrywidth{##2}}\@nameuse{LT@\roman{LT@tables}} \fi \endgroup}

% \setlength{\parindent}{0.5in}
% \setlength{\parskip}{0pt plus 0pt minus 0pt}

% Overwrite redefinition of paragraph and subparagraph by the default LaTeX template
% See https://github.com/crsh/papaja/issues/292
\makeatletter
\renewcommand{\paragraph}{\@startsection{paragraph}{4}{\parindent}%
  {0\baselineskip \@plus 0.2ex \@minus 0.2ex}%
  {-1em}%
  {\normalfont\normalsize\bfseries\itshape\typesectitle}}

\renewcommand{\subparagraph}[1]{\@startsection{subparagraph}{5}{1em}%
  {0\baselineskip \@plus 0.2ex \@minus 0.2ex}%
  {-\z@\relax}%
  {\normalfont\normalsize\itshape\hspace{\parindent}{#1}\textit{\addperi}}{\relax}}
\makeatother

% \usepackage{etoolbox}
\makeatletter
\patchcmd{\HyOrg@maketitle}
  {\section{\normalfont\normalsize\abstractname}}
  {\section*{\normalfont\normalsize\abstractname}}
  {}{\typeout{Failed to patch abstract.}}
\patchcmd{\HyOrg@maketitle}
  {\section{\protect\normalfont{\@title}}}
  {\section*{\protect\normalfont{\@title}}}
  {}{\typeout{Failed to patch title.}}
\makeatother

\usepackage{xpatch}
\makeatletter
\xapptocmd\appendix
  {\xapptocmd\section
    {\addcontentsline{toc}{section}{\appendixname\ifoneappendix\else~\theappendix\fi\\: #1}}
    {}{\InnerPatchFailed}%
  }
{}{\PatchFailed}
\keywords{keywords\newline\indent Word count: X}
\DeclareDelayedFloatFlavor{ThreePartTable}{table}
\DeclareDelayedFloatFlavor{lltable}{table}
\DeclareDelayedFloatFlavor*{longtable}{table}
\makeatletter
\renewcommand{\efloat@iwrite}[1]{\immediate\expandafter\protected@write\csname efloat@post#1\endcsname{}}
\makeatother
\usepackage{csquotes}
\ifLuaTeX
  \usepackage{selnolig}  % disable illegal ligatures
\fi
\IfFileExists{bookmark.sty}{\usepackage{bookmark}}{\usepackage{hyperref}}
\IfFileExists{xurl.sty}{\usepackage{xurl}}{} % add URL line breaks if available
\urlstyle{same} % disable monospaced font for URLs
\hypersetup{
  pdftitle={Supplementary Material 1},
  pdflang={en-EN},
  pdfkeywords={keywords},
  hidelinks,
  pdfcreator={LaTeX via pandoc}}

\title{Supplementary Material 1}
\author{\phantom{0}}
\date{}


\shorttitle{Supplementary Material 1}

\affiliation{\phantom{0}}

\begin{document}
\maketitle

\hypertarget{inter-rater-reliability-methods}{%
\section{Inter-rater reliability methods}\label{inter-rater-reliability-methods}}

As the Cohen Kappa resulted in suspicious values during the first step of the agreement analysis, we have decided to use other inter reliability tests to examine degree of agreement. It was possible to use Matthews correlation coefficient or other methods such as the Gwet´s AC1. The later was chosen because it overcomes problems associated with the Cohen´s Kappa if the degree of agreement is in fact high (Gwet, 2008). Moreover, the Gwet´s AC1 it also does not have an assumptions, which are sometimes difficult to fulfill e.g.~independence between raters (Gwet, 2008). The text below presents statistical calculations conducted in the present study to evaluate the degree of agreement between raters.

\newpage

\begin{Shaded}
\begin{Highlighting}[]
\NormalTok{data.abs.scr }\OtherTok{\textless{}{-}}\NormalTok{ openxlsx}\SpecialCharTok{::}\FunctionTok{read.xlsx}\NormalTok{(}
  \FunctionTok{paste0}\NormalTok{(}\FunctionTok{getwd}\NormalTok{(),}\StringTok{"/Data/R\_data\_study\_selection.xlsx"}\NormalTok{))}
\NormalTok{data.qual.scr }\OtherTok{\textless{}{-}}\NormalTok{ openxlsx}\SpecialCharTok{::}\FunctionTok{read.xlsx}\NormalTok{(}
  \FunctionTok{paste0}\NormalTok{(}\FunctionTok{getwd}\NormalTok{(),}\StringTok{"/Data/R\_data\_quality\_assessment.xlsx"}\NormalTok{))}
\end{Highlighting}
\end{Shaded}

\begin{Shaded}
\begin{Highlighting}[]
\CommentTok{\# set seed in order to assure computaional reproducitiblity}
\FunctionTok{set.seed}\NormalTok{(}\DecValTok{874354}\NormalTok{)}

\CommentTok{\# percent ageement}
\FunctionTok{agree}\NormalTok{(data.abs.scr, }\AttributeTok{tolerance=}\DecValTok{0}\NormalTok{) }
\end{Highlighting}
\end{Shaded}

\begin{verbatim}
##  Percentage agreement (Tolerance=0)
## 
##  Subjects = 627 
##    Raters = 2 
##   %-agree = 93.3
\end{verbatim}

\begin{Shaded}
\begin{Highlighting}[]
\CommentTok{\# cohens kappa}
\CommentTok{\# psych package}
\NormalTok{psych}\SpecialCharTok{::}\FunctionTok{cohen.kappa}\NormalTok{(data.abs.scr)}
\end{Highlighting}
\end{Shaded}

\begin{verbatim}
## Call: cohen.kappa1(x = x, w = w, n.obs = n.obs, alpha = alpha, levels = levels)
## 
## Cohen Kappa and Weighted Kappa correlation coefficients and confidence boundaries 
##                  lower estimate upper
## unweighted kappa  0.11     0.25   0.4
## weighted kappa    0.11     0.25   0.4
## 
##  Number of subjects = 627
\end{verbatim}

\begin{Shaded}
\begin{Highlighting}[]
\CommentTok{\# irr package}
\CommentTok{\# kappa2(ratings = data.abs.scr) \# cohens kappa yields}
\CommentTok{\#contraceptive values, thus other methods }
\CommentTok{\#such as. Matthews correlation coefficient}
\CommentTok{\#or Gwet´s AC1 migt be used for further analysis. }

\CommentTok{\# Matthews correlation coefficient (dodat zdůvodnění proč zrovna toto)}
\CommentTok{\# mltools::mcc(data.abs.scr$PM, data.abs.scr$JH) \%\textgreater{}\% round(digits = 2) }

\CommentTok{\# Gwet´s AC1}
\NormalTok{gwen}\OtherTok{=}\NormalTok{irrCAC}\SpecialCharTok{::}\FunctionTok{gwet.ac1.raw}\NormalTok{(}\AttributeTok{ratings =}\NormalTok{ data.abs.scr)}
\NormalTok{gwen}
\end{Highlighting}
\end{Shaded}

\begin{verbatim}
## $est
##   coeff.name        pa         pe coeff.val coeff.se      conf.int p.value
## 1        AC1 0.9330144 0.08822549   0.92653  0.01169 (0.904,0.949)       0
##       w.name
## 1 unweighted
## 
## $weights
##      [,1] [,2]
## [1,]    1    0
## [2,]    0    1
## 
## $categories
## [1] 0 1
\end{verbatim}

\begin{Shaded}
\begin{Highlighting}[]
\CommentTok{\# percent ageement}
\FunctionTok{agree}\NormalTok{(data.qual.scr, }\AttributeTok{tolerance=}\DecValTok{0}\NormalTok{) }\CommentTok{\# 96\%}
\end{Highlighting}
\end{Shaded}

\begin{verbatim}
##  Percentage agreement (Tolerance=0)
## 
##  Subjects = 143 
##    Raters = 2 
##   %-agree = 95.8
\end{verbatim}

\begin{Shaded}
\begin{Highlighting}[]
\CommentTok{\# cohens kappa}
\CommentTok{\#...............}
\CommentTok{\# psych package}
\NormalTok{psych}\SpecialCharTok{::}\FunctionTok{cohen.kappa}\NormalTok{(data.qual.scr) }\CommentTok{\# 0.9}
\end{Highlighting}
\end{Shaded}

\begin{verbatim}
## Call: cohen.kappa1(x = x, w = w, n.obs = n.obs, alpha = alpha, levels = levels)
## 
## Cohen Kappa and Weighted Kappa correlation coefficients and confidence boundaries 
##                  lower estimate upper
## unweighted kappa  0.82      0.9  0.98
## weighted kappa    0.82      0.9  0.98
## 
##  Number of subjects = 143
\end{verbatim}

\begin{Shaded}
\begin{Highlighting}[]
\CommentTok{\# irr package}
\FunctionTok{kappa2}\NormalTok{(}\AttributeTok{ratings =}\NormalTok{ data.qual.scr) }\CommentTok{\# 0.9}
\end{Highlighting}
\end{Shaded}

\begin{verbatim}
##  Cohen's Kappa for 2 Raters (Weights: unweighted)
## 
##  Subjects = 143 
##    Raters = 2 
##     Kappa = 0.9 
## 
##         z = 10.8 
##   p-value = 0
\end{verbatim}

\begin{Shaded}
\begin{Highlighting}[]
\CommentTok{\# Gwet´s AC1}
\NormalTok{gwen}\OtherTok{=}\NormalTok{irrCAC}\SpecialCharTok{::}\FunctionTok{gwet.ac1.raw}\NormalTok{(}\AttributeTok{ratings =}\NormalTok{ data.qual.scr) }\CommentTok{\# 0.93}
\NormalTok{gwen}
\end{Highlighting}
\end{Shaded}

\begin{verbatim}
## $est
##   coeff.name       pa        pe coeff.val coeff.se      conf.int p.value
## 1        AC1 0.958042 0.4205585   0.92759  0.02962 (0.869,0.986)       0
##       w.name
## 1 unweighted
## 
## $weights
##      [,1] [,2]
## [1,]    1    0
## [2,]    0    1
## 
## $categories
## [1] 0 1
\end{verbatim}

\newpage

\hypertarget{references}{%
\section{References}\label{references}}

\begingroup
\setlength{\parindent}{-0.5in}
\setlength{\leftskip}{0.5in}

\hypertarget{refs}{}
\begin{CSLReferences}{1}{0}
\leavevmode\vadjust pre{\hypertarget{ref-Gwet_2008}{}}%
Gwet, K. L. (2008). Computing inter-rater reliability and its variance in the presence of high agreement. \emph{British Journal of Mathematical and Statistical Psychology}, \emph{61}(1), 29--48. \url{https://doi.org/10.1348/000711006X126600}

\end{CSLReferences}

\endgroup


\end{document}
